\documentclass[a4j]{jarticle}
\usepackage{alltt}
\usepackage{proof}
\usepackage[dvipdfmx]{graphicx,color}
\newcommand{\CAL}[1]{\mbox{${\cal #1}$}}
\newcommand{\drv}{\longrightarrow}
\newcommand{\lra}{\longrightarrow}
\newcommand{\imp}{\supset}

\title{プログラミング言語特論レポート(サンプル集)}
\author{情知 太郎 (012x356x)}
\date{平成29年XX月YY日}
\begin{document}
\maketitle

%%%%%%%%%%%%%%%%%%%%%%%%%%%%%%%%%%%%%%%%%%%%%%%%%%%%%%%%%%%%%%%%%%%%%%%
\section{論理式の書き方}

\begin{itemize}
\item $p(x)\land q(x)$
\begin{verbatim}
  $p(x)\land q(x)$
\end{verbatim}
\item $p(x)\imp q(x)$
\begin{verbatim}
  $p(x)\imp q(x)$
\end{verbatim}
\item $\forall x. p(x)$
\begin{verbatim}
  $\forall x. p(x)$
\end{verbatim}
\item $\exists x. p(x)$
\begin{verbatim}
  $\exists x. p(x)$
\end{verbatim}
\end{itemize}
%%%%%%%%%%%%%%%%%%%%%%%%%%%%%%%%%%%%%%%%%%%%%%%%%%%%%%%%%%%%%%%%%%%%%%%
\section{証明図の書き方}
証明図を書くために,\texttt{proof.sty}を使います.
\begin{itemize}
\item 一般形
\begin{verbatim}
\[
 \infer[規則1]{A_1, \ldots , A_n}{B_1,\ldots , B_m}
\]
\end{verbatim}
と書くと,以下のように表示されます.
\[
 \infer[規則1]{A_1, \ldots , A_n}{B_1,\ldots , B_m}
\]
\item $\land$R規則
\begin{verbatim}
\[
\infer[\land R]{\CAL{P}\drv G_1\land G_2}{
  \CAL{P}\drv G_1 & \CAL{P}\drv G_2
}
\]
\end{verbatim}
と書くと,以下のように表示されます.
\[
\infer[\land R]{\CAL{P}\drv G_1\land G_2}{
  \CAL{P}\drv G_1 & \CAL{P}\drv G_2
}
\]
\item $decide$規則
\begin{verbatim}
\[
\infer[decide]{\CAL{P}\drv A}{
  \CAL{P}\stackrel{D}{\drv} A
}
\]
\end{verbatim}
と書くと,以下のように表示されます.
\[
\infer[decide]{\CAL{P}\drv A}{
  \CAL{P}\stackrel{D}{\drv} A
}
\]
\end{itemize}
%%%%%%%%%%%%%%%%%%%%%%%%%%%%%%%%%%%%%%%%%%%%%%%%%%%%%%%%%%%%%%%%%%%%%%%
\section{Prologと証明図 }

\begin{itemize}
\item 以下はPrologプログラムの例です.
\begin{alltt}
  % プログラム
  app([], Y, Y).
  app([X|Xs], Y, [X|Zs]) :- app(Xs, Y, Zs).

  % ゴール
  ?- app([1,2], [3,4], W).
  W = [1, 2, 3, 4] 
  Yes
\end{alltt}

\item 以下はPrologプログラムのトレース例です.
\begin{alltt}
  ?- trace.
  Yes
  [trace]  ?- app([1,2], [3,4], W).
     {\color{red}Call: (7)} app([1, 2], [3, 4], _G316) ? 
     {\color{green}Call: (8)} app([2], [3, 4], _G379) ? 
     {\color{blue}Call: (9)} app([], [3, 4], _G382) ? 
     {\color{blue}Exit: (9)} app([], [3, 4], [3, 4]) ? 
     {\color{green}Exit: (8)} app([2], [3, 4], [2, 3, 4]) ? 
     {\color{red}Exit: (7)} app([1, 2], [3, 4], [1, 2, 3, 4]) ? 
  X = [1, 2, 3, 4] 
  Yes
\end{alltt}

\item 以下は,プログラムとゴールを論理式で表した例です.
\begin{alltt}
% プログラム
app([], Y, Y).
app([X|Xs], Y, [X|Zs]) :- app(Xs, Y, Zs).
\end{alltt}
\qquad\qquad $\Downarrow$
\begin{flushleft}{\color{red}
\(\forall y.app([], y, y) \) \\
\(\forall x,x_s,y,z_s.(app(x_s,y,z_s)\imp app([x|x_s],y,[x|z_s]) \) }
\end{flushleft}

\begin{alltt}
% ゴール
?- app([1,2], [3,4], W).
\end{alltt}
\qquad\qquad$\Downarrow$
\begin{flushleft}{\color{red}
\(\exists w.app([1,2],[3,4],w)\)} 
\end{flushleft}
\end{itemize}

\end{document}
